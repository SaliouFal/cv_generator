\documentclass[a4paper]{article}

%%%%%%%%%%%%%%%%%%%%%%%%%%%%%%%%%%%%%%%%%%%%%%
\usepackage[T1]{fontenc}
\usepackage{geometry}
\geometry{a4paper,left=1.5cm,right=1cm,top=1cm,bottom=1cm}
\usepackage[utf8]{inputenc}
\usepackage{graphicx}
\usepackage[absolute,overlay]{textpos}
% \usepackage{eso-pic}               % image de fond – supprimé car plus utilisé
\usepackage{textcomp}
\usepackage{newtxtext}
\usepackage{fontawesome5}
\usepackage[hidelinks]{hyperref}
\usepackage{tikz}
\usepackage{xcolor}
\usepackage{enumitem}
\setlist{nosep,leftmargin=6mm}
\usepackage{times}                % même police que votre exemple
\usepackage{array}
\usepackage{tabularx}
\usepackage{ragged2e}
\let\origcolorbox\colorbox    % sauvegarde
\renewcommand{\colorbox}[2]{#2}% neutralise le fond

%%%%%%%%%%%%%%%%%%%%%%%%%%%%%%%%%%%%%%%%%%%%%%
% Définition d’un gris très clair (~95 % blanc)
\definecolor{verylightgray}{HTML}{F5F5F5}

%%%%%%%%%%%%%%%%%%%%%%%%%%%%%%%%%%%%%%%%%%
\pagecolor{verylightgray}   % fond gris très clair
% le texte est noir par défaut ; aucune redéfinition nécessaire

%%%%%%%%%%%%%%%%%%%%%%%%%%%%%%%%%%%%%%%%%%%%%%
%\pagecolor{lightgray}     % fond gris clair
\color{black}             % texte en noir

% Vous pouvez ajuster la nuance si besoin : par ex. \pagecolor{gray!15}

\providecolor{sidetext}{rgb}{1,1,1}
\definecolor{maincolor}{HTML}{ffffff}

%%%%%%%%%%%%%%%%%%%%%%%%%%%%%%%%%%%%%%%%
\newcommand{\fullrule}{\hspace{-1.5cm}\rule{\paperwidth}{0.4pt}}
\newcommand{\cvsection}[1]{%
  \vspace{6pt}\textbf{\Large #1}\par\vspace{2pt}}
\newcommand{\cicon}[1]{%
  \tikz[baseline]{\draw[fill=white] (0,0.1) circle[radius=0.1cm];}~#1}

\setlength{\parindent}{0pt}
%%%%%%%%%%%%%%%%%%%%%%%%%%%%%%%%%%%%%%%%%%%%%%%%%%%%%%%%%%%%%%
\begin{document}

% ---------- Photo ------------------------------------------------
\ifx\relax\relax\else
\begin{textblock*}{4cm}(0.2cm,0.3cm)
  \includegraphics[width=2.5cm,clip,keepaspectratio]{}
\end{textblock*}
 \fi
% ---------- En-tête ---------------------------------------------
\begin{center}
  {\fontsize{44pt}{24pt}\selectfont\bfseries Saoane Leon}

  \bigskip
  {\Large Responsable RSE | Stratégie \& Impact durable}

  \bigskip\bigskip
  \faMapMarker~8 rue Frank, Paris, France\ 
  \quad\faEnvelope~\href{mailto:ssssssss@see.fr}{ssssssss@see.fr}

  \bigskip
  % Badge LinkedIn (retirez-le si inutile)
  \faPhone~ +33 6 00 00 00 00
  \quad \faLinkedin\ \href{linkedin.com/in/saoaneleon}{saoaneleon}
 

  %\vspace{-0.3cm}
  
\end{center}
%\medskip\fullrule
 \vspace{0.6cm}
% ---------- Profil ----------------------------------------------
\cvsection{Profil}
\vspace{0.3cm}
Professionnelle engagée dans la responsabilité sociétale, expérimentée du conseil à la direction d’équipes RSE. Spécialisée dans la définition de KPI, la cartographie des risques et l’évaluation d’impact social. Solides compétences en gestion de projet en environnements internationaux et multiculturels. Animée par l’amélioration continue et la création de valeur durable pour les organisations.

\medskip\fullrule

% ---------- Expérience ------------------------------------------
\cvsection{Expérience}
\vspace{0.3cm}
\colorbox{maincolor}{%
  \begin{minipage}{\linewidth}
    \noindent
    \textbf{Responsable RSE}\hfill 01/2032 - 12/2035\\
    La Rosée\\[-0.3em]
    \begin{itemize}[leftmargin=*]
      \item Déployé des programmes de sensibilisation RSE, renforçant l’engagement interne. \item Défini des KPI et piloté des plans d’amélioration continue pour atteindre les objectifs durables.
    \end{itemize}
  \end{minipage}}

\vspace{3mm}

\colorbox{maincolor}{%
  \begin{minipage}{\linewidth}
    \noindent
    \textbf{Chargé de projet RSE}\hfill 01/2030 - 12/2032\\
    La Rosée\\[-0.3em]
    \begin{itemize}[leftmargin=*]
      \item Actualisé la cartographie des risques et conduit des audits internes conformes aux normes. \item Formé les collaborateurs et assuré le reporting réglementaire tout en effectuant une veille légale.
    \end{itemize}
  \end{minipage}}

\vspace{3mm}

\colorbox{maincolor}{%
  \begin{minipage}{\linewidth}
    \noindent
    \textbf{Consultante junior}\hfill 01/2028 - 12/2030\\
    Kimso\\[-0.3em]
    \begin{itemize}[leftmargin=*]
      \item Accompagné associations et fondations dans la structuration de projets à impact social. \item Évalué l’impact et recommandé des actions stratégiques pour accroître la valeur sociétale.
    \end{itemize}
  \end{minipage}}

\vspace{3mm}

\colorbox{maincolor}{%
  \begin{minipage}{\linewidth}
    \noindent
    \textbf{Stagiaire Consultante}\hfill 05/2028 - 10/2028\\
    Kimso\\[-0.3em]
    \begin{itemize}[leftmargin=*]
      \item Réalisé une étude d’impact social complète pour un client associatif. \item Proposé des recommandations opérationnelles améliorant la performance sociétale.
    \end{itemize}
  \end{minipage}}

\vspace{3mm}

\colorbox{maincolor}{%
  \begin{minipage}{\linewidth}
    \noindent
    \textbf{Stagiaire Assistant Social Brand Manager}\hfill 07/2026 - 12/2026\\
    L’Oréal\\[-0.3em]
    \begin{itemize}[leftmargin=*]
      \item Conçu et suivi des projets sociaux intégrés à la stratégie de marque. \item Valorisé les initiatives auprès des parties prenantes, renforçant la visibilité RSE.
    \end{itemize}
  \end{minipage}}

\medskip\fullrule

% ---------- Éducation -------------------------------------------
\cvsection{Éducation}
\vspace{0.3cm}

\colorbox{maincolor}{%
  \begin{minipage}{\linewidth}
    \noindent
    \textbf{MSc Project Management for Business}\hfill 09/2027 - 06/2028\\
    SKEMA Business School – Campus Paris / Belo Horizonte / Sophia Antipolis\\[-0.3em]
    \begin{itemize}[leftmargin=*]
      \item Méthodes avancées de gestion de projet et planification stratégique. \item Gestion des risques, contrôle des coûts et leadership d’équipe.
    \end{itemize}
  \end{minipage}}

\vspace{3mm}

\colorbox{maincolor}{%
  \begin{minipage}{\linewidth}
    \noindent
    \textbf{Licence d’Économie et de Gestion Internationale}\hfill 09/2022 - 06/2025\\
    Université Catholique de Lille\\[-0.3em]
    \begin{itemize}[leftmargin=*]
      \item Analyse macro et microéconomique appliquée aux marchés mondiaux. \item Finance internationale et management des organisations.
    \end{itemize}
  \end{minipage}}

\vspace{3mm}

\colorbox{maincolor}{%
  \begin{minipage}{\linewidth}
    \noindent
    \textbf{Baccalauréat Général – Spécialités Mathématiques et SES}\hfill 09/2021 - 06/2022\\
    Lycée La Favorite, Lyon\\[-0.3em]
    \begin{itemize}[leftmargin=*]
      \item Approfondissement des fondamentaux en mathématiques et sciences économiques.
    \end{itemize}
  \end{minipage}}

\medskip\fullrule

% ---------- Compétences -----------------------------------------
\cvsection{Compétences}
\vspace{0.3cm}

\begin{tabular}{@{}p{0.25\linewidth}p{0.18\linewidth}p{0.18\linewidth}p{0.18\linewidth}}\cicon RSE & \cicon Gestion & \cicon Audit & \cicon Reporting \\
\cicon Stratégie & \cicon Risque & \cicon Évaluation & \cicon Logistique \\
\cicon Leadership & \cicon Communication & ~ & ~ \\\end{tabular}   % grille 3 lignes × 4 colonnes

\end{document}

\documentclass[a4paper]{article}

%%%%%%%%%%%%%%%%%%%%%%%%%%%%%%%%%%%%%%%%%%%%%%
\usepackage[T1]{fontenc}
\usepackage{geometry}
\geometry{a4paper,left=1.5cm,right=1cm,top=1cm,bottom=1cm}
\usepackage[utf8]{inputenc}
\usepackage{graphicx}
\usepackage[absolute,overlay]{textpos}
% \usepackage{eso-pic}               % image de fond – supprimé car plus utilisé
\usepackage{textcomp}
\usepackage{newtxtext}
\usepackage{fontawesome5}
\usepackage[hidelinks]{hyperref}
\usepackage{tikz}
\usepackage{xcolor}
\usepackage{enumitem}
\setlist{nosep,leftmargin=6mm}
\usepackage{times}                % même police que votre exemple
\usepackage{array}
\usepackage{tabularx}
\usepackage{ragged2e}
\let\origcolorbox\colorbox    % sauvegarde
\renewcommand{\colorbox}[2]{#2}% neutralise le fond

%%%%%%%%%%%%%%%%%%%%%%%%%%%%%%%%%%%%%%%%%%%%%%
% Définition d’un gris très clair (~95 % blanc)
\definecolor{verylightgray}{HTML}{F5F5F5}

%%%%%%%%%%%%%%%%%%%%%%%%%%%%%%%%%%%%%%%%%%
\pagecolor{verylightgray}   % fond gris très clair
% le texte est noir par défaut ; aucune redéfinition nécessaire

%%%%%%%%%%%%%%%%%%%%%%%%%%%%%%%%%%%%%%%%%%%%%%
%\pagecolor{lightgray}     % fond gris clair
\color{black}             % texte en noir

% Vous pouvez ajuster la nuance si besoin : par ex. \pagecolor{gray!15}

\providecolor{sidetext}{rgb}{1,1,1}
\definecolor{maincolor}{HTML}{ffffff}

%%%%%%%%%%%%%%%%%%%%%%%%%%%%%%%%%%%%%%%%
\newcommand{\fullrule}{\hspace{-1.5cm}\rule{\paperwidth}{0.4pt}}
\newcommand{\cvsection}[1]{%
  \vspace{6pt}\textbf{\Large #1}\par\vspace{2pt}}
\newcommand{\cicon}[1]{%
  \tikz[baseline]{\draw[fill=white] (0,0.1) circle[radius=0.1cm];}~#1}

\setlength{\parindent}{0pt}
%%%%%%%%%%%%%%%%%%%%%%%%%%%%%%%%%%%%%%%%%%%%%%%%%%%%%%%%%%%%%%
\begin{document}

% ---------- Photo ------------------------------------------------
\ifx\relax\relax\else
\begin{textblock*}{4cm}(0.2cm,0.3cm)
  \includegraphics[width=2.5cm,clip,keepaspectratio]{}
\end{textblock*}
 \fi
% ---------- En-tête ---------------------------------------------
\begin{center}
  {\fontsize{44pt}{24pt}\selectfont\bfseries Saoane Leon}

  \bigskip
  {\Large Responsable RSE – Stratégie \& Performance Durable}

  \bigskip\bigskip
  \faMapMarker~8 rue Frank, Paris, France\ 
  \quad\faEnvelope~\href{mailto:ssssssss@see.fr}{ssssssss@see.fr}

  \bigskip
  % Badge LinkedIn (retirez-le si inutile)
  \faPhone~ +33 6 00 00 00 00
  \quad \faLinkedin\ \href{Syno Sionnnne llourye}{Syno Sionnnne llourye}
 

  %\vspace{-0.3cm}
  
\end{center}
%\medskip\fullrule
 \vspace{0.6cm}
% ---------- Profil ----------------------------------------------
\cvsection{Profil}
\vspace{0.3cm}
Professionnelle engagée dans la responsabilité sociétale, j’ai piloté des projets RSE et défini des indicateurs de performance pour des entreprises à impact. Mon parcours associe management de projet, audits de conformité et accompagnement du changement. Formée dans des environnements internationaux, je maîtrise la gestion des risques et la mesure d’impact social. Orientée résultats, je souhaite renforcer la durabilité et la performance globale des organisations.

\medskip\fullrule

% ---------- Expérience ------------------------------------------
\cvsection{Expérience}
\vspace{0.3cm}
\colorbox{maincolor}{%
  \begin{minipage}{\linewidth}
    \noindent
    \textbf{Responsable RSE}\hfill 01/2032 - 12/2035\\
    La Rosée\\[-0.3em]
    \begin{itemize}[leftmargin=*]
      \item Animé des sessions de sensibilisation RSE pour l’ensemble des collaborateurs, renforçant l’engagement interne. \item Défini des indicateurs clés et recommandé des axes d’amélioration pour piloter la performance durable.
    \end{itemize}
  \end{minipage}}

\vspace{3mm}

\colorbox{maincolor}{%
  \begin{minipage}{\linewidth}
    \noindent
    \textbf{Chargée de projet RSE}\hfill 01/2030 - 12/2032\\
    La Rosée\\[-0.3em]
    \begin{itemize}[leftmargin=*]
      \item Actualisé la cartographie des risques et conduit des audits réguliers, réduisant les non-conformités. \item Formé les équipes aux exigences RSE et assuré un reporting fiable auprès de la direction.
    \end{itemize}
  \end{minipage}}

\vspace{3mm}

\colorbox{maincolor}{%
  \begin{minipage}{\linewidth}
    \noindent
    \textbf{Consultante junior}\hfill 01/2028 - 12/2030\\
    Kimso\\[-0.3em]
    \begin{itemize}[leftmargin=*]
      \item Accompagné associations et fondations dans la structuration de projets, clarifiant objectifs et indicateurs d’impact. \item Réalisé des évaluations d’impact social et fourni des recommandations stratégiques.
    \end{itemize}
  \end{minipage}}

\vspace{3mm}

\colorbox{maincolor}{%
  \begin{minipage}{\linewidth}
    \noindent
    \textbf{Membre pôle humanitaire}\hfill 01/2025 - 12/2028\\
    Association Hope\\[-0.3em]
    \begin{itemize}[leftmargin=*]
      \item Organisé des actions humanitaires locales en coordonnant bénévoles et ressources. \item Supervisé la collecte de fonds et de matériel au bénéfice direct des communautés ciblées.
    \end{itemize}
  \end{minipage}}

\vspace{3mm}

\colorbox{maincolor}{%
  \begin{minipage}{\linewidth}
    \noindent
    \textbf{Responsable logistique}\hfill 09/2023 - 06/2025\\
    Bureau des Sports – Université Catholique de Lille\\[-0.3em]
    \begin{itemize}[leftmargin=*]
      \item Planifié la logistique d’événements sportifs étudiants réunissant plus de 300 participants. \item Géré budget, fournisseurs et plannings, optimisant coûts et qualité.
    \end{itemize}
  \end{minipage}}

\medskip\fullrule

% ---------- Éducation -------------------------------------------
\cvsection{Éducation}
\vspace{0.3cm}

\colorbox{maincolor}{%
  \begin{minipage}{\linewidth}
    \noindent
    \textbf{MSc Project Management for Business}\hfill 09/2027 - 06/2028\\
    SKEMA Business School – Campus de Paris / Belo Horizonte / Sophia Antipolis\\[-0.3em]
    \begin{itemize}[leftmargin=*]
      \item Gestion de projet avancée (méthodes agiles, pilotage de portefeuille) et management interculturel.
    \end{itemize}
  \end{minipage}}

\vspace{3mm}

\colorbox{maincolor}{%
  \begin{minipage}{\linewidth}
    \noindent
    \textbf{Licence d’Économie et de Gestion Internationale}\hfill 09/2022 - 06/2025\\
    Université Catholique de Lille\\[-0.3em]
    \begin{itemize}[leftmargin=*]
      \item Microéconomie, macroéconomie, finance d’entreprise et commerce international.
    \end{itemize}
  \end{minipage}}

\vspace{3mm}

\colorbox{maincolor}{%
  \begin{minipage}{\linewidth}
    \noindent
    \textbf{Baccalauréat général (Mathématiques, SES) – Mention Bien}\hfill 09/2021 - 06/2022\\
    Lycée La Favorite, Lyon\\[-0.3em]
    \begin{itemize}[leftmargin=*]
      \item Mathématiques avancées et sciences économiques et sociales.
    \end{itemize}
  \end{minipage}}

\medskip\fullrule

% ---------- Compétences -----------------------------------------
\cvsection{Compétences}
\vspace{0.3cm}

\begin{tabular}{@{}p{0.25\linewidth}p{0.18\linewidth}p{0.18\linewidth}p{0.18\linewidth}}\cicon Gestion & \cicon Audit & \cicon Reporting & \cicon Analyse \\
\cicon Leadership & \cicon Communication & \cicon RSE & \cicon Gestion des risques \\
\cicon Planification & \cicon Coordination & ~ & ~ \\\end{tabular}   % grille 3 lignes × 4 colonnes

\end{document}

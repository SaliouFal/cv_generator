\documentclass[a4paper]{article}

%%%%%%%%%%%%%%%%%%%%%%%%%%%%%%%%%%%%%%%%%%%%%%
\usepackage[T1]{fontenc}
\usepackage{geometry}
\geometry{a4paper,left=1.5cm,right=1cm,top=1cm,bottom=1cm}

\usepackage{graphicx}
\usepackage[absolute,overlay]{textpos}
\usepackage{eso-pic}               % image de fond
\usepackage{fontawesome5}
\usepackage[hidelinks]{hyperref}
\usepackage{tikz}
\usepackage{xcolor}
\usepackage{enumitem}
\setlist{nosep,leftmargin=6mm}
\usepackage{times}                % même police que votre exemple
\usepackage{array} 
\usepackage{tabularx}
\usepackage{ragged2e}
\let\origcolorbox\colorbox    % sauvegarde
\renewcommand{\colorbox}[2]{#2}% neutralise le fond
%%%%%%%%%%%%%%%%%%%%%%%%%%%%%%%%%%%%%%%%%%%%%%
%\definecolor{texcolor}{HTML}{e2e8f0}
\providecolor{sidetext}{rgb}{1,1,1}
\definecolor{maincolor}{HTML}{ffffff}

%%%%%%%%%%%%%%%%%%%%%%%%%%%%%%%%%%%%%%%%%
% — Ne changez pas le nom : « background.jpg » doit être présent
\AddToShipoutPictureBG*{%
  \includegraphics[width=\paperwidth,height=\paperheight]{background.jpg}%
}

%%%%%%%%%%%%%%%%%%%%%%%%%%%%%%%%%%%%%%%%%
\newcommand{\fullrule}{\hspace{-1.5cm}\rule{\paperwidth}{0.4pt}}
\newcommand{\cvsection}[1]{%
  \vspace{6pt}\textbf{\Large #1}\par\vspace{2pt}}
\newcommand{\cicon}[1]{%
  \tikz[baseline]{\draw[fill=white] (0,0.1) circle[radius=0.1cm];}~#1}

\setlength{\parindent}{0pt}
%\color{texcolor}
%%%%%%%%%%%%%%%%%%%%%%%%%%%%%%%%%%%%%%%%%%%%%%%%%%%%%%%%%%%%%%
\begin{document}
\color{white}
% ---------- Photo ------------------------------------------------
\begin{textblock*}{4cm}(0.2cm,0.3cm)
  \includegraphics[width=2.5cm,clip,keepaspectratio]{90c8a84018ad4dd2828c474a6a961064.png}
\end{textblock*}

% ---------- En-tête ---------------------------------------------
\begin{center}
  {\fontsize{44pt}{24pt}\selectfont\bfseries Judikael Mourouvin}

  \bigskip
  {\Large Technicien Informatique \& Marketing digital}

  \bigskip\bigskip
  \faMapMarker~Route de COCOYER\ 97190 GOSIER
  \quad\faEnvelope~\href{mailto:jkmou971@gmail.com}{jkmou971@gmail.com}

  \bigskip
  % Badge LinkedIn (retirez-le si inutile)
  \faPhone~ +590 0690 91 14 48
  \quad \faLinkedin\ \href{}{}
 

  \vspace{-0.3cm}
  \medskip\fullrule
\end{center}

% ---------- Profil ----------------------------------------------
\cvsection{Profil}

Passionné par l’informatique et le marketing digital, j’ai développé des compétences solides en maintenance, support utilisateur et gestion de projets numériques. Mon année d’alternance à la DSI de la Mairie du Gosier m’a permis de participer activement aux initiatives digitales de la collectivité. Curieux et rigoureux, j’assure la configuration des postes, le diagnostic des incidents et l’accompagnement des utilisateurs. Je souhaite désormais mettre mon expertise au service de nouvelles missions à plein temps.

\medskip\fullrule

% ---------- Expérience ------------------------------------------
\cvsection{Expérience}
\hspace*{1.3cm}%

\colorbox{maincolor}{%
  \begin{minipage}{\linewidth}
    \textbf{Alternant en Marketing Digital} \\ Mairie du Gosier – DSI \\ 2023-2024
    \begin{itemize}
      \item Piloté des projets numériques municipaux, assurant leur planification et leur suivi efficaces. \item Analysé les besoins des agents et déployé des solutions logicielles adaptées, améliorant la productivité. \item Assuré le support et formé plus de 40 utilisateurs, renforçant l’adoption des outils digitaux.
    \end{itemize}
  \end{minipage}}

\vspace{3mm}


\colorbox{maincolor}{%
  \begin{minipage}{\linewidth}
    \textbf{Animateur de la zone informatique} \\ Pôle Emploi, Gosier \\ 2022-2023
    \begin{itemize}
      \item Fournit un support de proximité aux demandeurs d’emploi, réduisant le temps de résolution des incidents. \item Installé et entretenu les postes informatiques de l’espace public, garantissant leur disponibilité constante. \item Diagnostiqué et corrigé les pannes matérielles et logicielles, assurant la continuité du service.
    \end{itemize}
  \end{minipage}}

\vspace{3mm}


\colorbox{maincolor}{%
  \begin{minipage}{\linewidth}
    \textbf{Stagiaire Informaticien} \\ NUMERIKA, Baie-Mahault \\ 2020-2021
    \begin{itemize}
      \item Configuré et maintenu les équipements (PC, imprimantes, réseau) du parc informatique. \item Répondu aux sollicitations des utilisateurs et documenté les solutions appliquées pour capitaliser les bonnes pratiques.
    \end{itemize}
  \end{minipage}}

\medskip\fullrule

% ---------- Éducation -------------------------------------------
\cvsection{Éducation}
\hspace*{1.3cm}%

    \begin{tabularx}{\linewidth}{@{}c >{\RaggedRight\arraybackslash}X@{}}
    \textcolor{sidetext}{\faGraduationCap} &
    \textbf{Bachelor Marketing Digital} \\
    & CFA IUTS \\
    & \textit{2023-2024} \\
    \end{tabularx}
    \begin{itemize}[leftmargin=*]
  \item Stratégies de marketing en ligne, SEO/SEA et gestion des réseaux sociaux.
  \item Gestion de projet digital et analyse de données marketing.
  \item Création de contenus et optimisation de la visibilité web.
\end{itemize}
\vspace{3mm}

    \begin{tabularx}{\linewidth}{@{}c >{\RaggedRight\arraybackslash}X@{}}
    \textcolor{sidetext}{\faGraduationCap} &
    \textbf{BTS Système Numérique option Informatique et Réseaux} \\
    & Lycée de Chevalier Saint Georges, Abymes \\
    & \textit{2019-2021} \\
    \end{tabularx}
    \begin{itemize}[leftmargin=*]
  \item Architecture des réseaux et administration des systèmes.
  \item Programmation, cybersécurité et maintenance matérielle.
  \item Diagnostic et résolution d’incidents informatiques.
\end{itemize}

\medskip\fullrule

% ---------- Compétences -----------------------------------------
\cvsection{Compétences}

\hspace*{2cm}%
\begin{tabular}{@{}p{0.25\linewidth}p{0.18\linewidth}p{0.18\linewidth}p{0.18\linewidth}}\cicon Administration & \cicon Maintenance & \cicon Réseaux & \cicon Assistance \\
\cicon Diagnostic & \cicon Configuration & \cicon Marketing & \cicon Support \\
\cicon Formation & \cicon Digital & ~ & ~ \\\end{tabular}   % grille 3 lignes × 4 colonnes

\end{document}

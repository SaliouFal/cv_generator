\documentclass[a4paper]{article}

%%%%%%%%%%%%%%%%%%%%%%%%%%%%%%%%%%%%%%%%%%%%%%
\usepackage[T1]{fontenc}
\usepackage{geometry}
\geometry{a4paper,left=1.5cm,right=1cm,top=1cm,bottom=1cm}

\usepackage{graphicx}
\usepackage[absolute,overlay]{textpos}
\usepackage{eso-pic}               % image de fond
\usepackage{fontawesome5}
\usepackage[hidelinks]{hyperref}
\usepackage{tikz}
\usepackage{xcolor}
\usepackage{enumitem}
\setlist{nosep,leftmargin=6mm}
\usepackage{times}                % même police que votre exemple
\usepackage{array} 
\usepackage{tabularx}
\usepackage{ragged2e}
\let\origcolorbox\colorbox    % sauvegarde
\renewcommand{\colorbox}[2]{#2}% neutralise le fond
%%%%%%%%%%%%%%%%%%%%%%%%%%%%%%%%%%%%%%%%%%%%%%
%\definecolor{texcolor}{HTML}{e2e8f0}
\providecolor{sidetext}{rgb}{1,1,1}
\definecolor{maincolor}{HTML}{ffffff}

%%%%%%%%%%%%%%%%%%%%%%%%%%%%%%%%%%%%%%%%%
% — Ne changez pas le nom : « background.jpg » doit être présent
\AddToShipoutPictureBG*{%
  \includegraphics[width=\paperwidth,height=\paperheight]{background.jpg}%
}

%%%%%%%%%%%%%%%%%%%%%%%%%%%%%%%%%%%%%%%%%
\newcommand{\fullrule}{\hspace{-1.5cm}\rule{\paperwidth}{0.4pt}}
\newcommand{\cvsection}[1]{%
  \vspace{6pt}\textbf{\Large #1}\par\vspace{2pt}}
\newcommand{\cicon}[1]{%
  \tikz[baseline]{\draw[fill=white] (0,0.1) circle[radius=0.1cm];}~#1}

\setlength{\parindent}{0pt}
%\color{texcolor}
%%%%%%%%%%%%%%%%%%%%%%%%%%%%%%%%%%%%%%%%%%%%%%%%%%%%%%%%%%%%%%
\begin{document}
\color{white}
% ---------- Photo ------------------------------------------------
\ifx\relax\relax\else
\begin{textblock*}{4cm}(0.2cm,0.3cm)
  \includegraphics[width=2.5cm,clip,keepaspectratio]{}
\end{textblock*}
 \fi
% ---------- En-tête ---------------------------------------------
\begin{center}
  {\fontsize{44pt}{24pt}\selectfont\bfseries Judikael MOUROUVIN}

  \bigskip
  {\Large Technicien informatique et marketing digital}

  \bigskip\bigskip
  \faMapMarker~Route de Cocoyer\ 97190 Le Gosier
  \quad\faEnvelope~\href{mailto:jkmou971@gmail.com}{jkmou971@gmail.com}

  \bigskip
  % Badge LinkedIn (retirez-le si inutile)
  \faPhone~ +590 0690 91 14 48
  \quad \faLinkedin\ \href{}{}
 

  \vspace{-0.3cm}
  
\end{center}
%\medskip\fullrule
% ---------- Profil ----------------------------------------------
\cvsection{Profil}

Passionné par l’informatique et le marketing digital, je maîtrise l’installation de postes, la maintenance préventive, le diagnostic d’incidents et l’accompagnement des utilisateurs. Mon alternance à la DSI de la mairie du Gosier m’a permis de piloter des projets numériques tout en formant les agents. Rigoureux et orienté service, je souhaite désormais évoluer en CDI pour soutenir vos initiatives digitales dans des environnements techniques variés.

\medskip\fullrule

% ---------- Expérience ------------------------------------------
\cvsection{Expérience}
\colorbox{maincolor}{%
  \begin{minipage}{\linewidth}
    \noindent
    \textbf{Alternant en marketing digital}\hfill 09/2023 - 08/2024\\
    Mairie du Gosier – DSI\\[-0.3em]
    \begin{itemize}[leftmargin=*]
      \item Participer à la conduite de projets numériques municipaux (site web, outils collaboratifs). \item Analyser les besoins des agents et déployer des solutions adaptées améliorant l’expérience utilisateur. \item Assurer support et formation pour renforcer l’adoption des outils digitaux.
    \end{itemize}
  \end{minipage}}

\vspace{3mm}

\colorbox{maincolor}{%
  \begin{minipage}{\linewidth}
    \noindent
    \textbf{Animateur de la zone informatique}\hfill 09/2022 - 08/2023\\
    Pôle emploi – Le Gosier\\[-0.3em]
    \begin{itemize}[leftmargin=*]
      \item Assister quotidiennement les demandeurs d’emploi sur l’utilisation des postes et services en ligne. \item Configurer et maintenir les postes de travail afin de garantir leur disponibilité. \item Diagnostiquer et résoudre les incidents pour réduire les interruptions de service.
    \end{itemize}
  \end{minipage}}

\vspace{3mm}

\colorbox{maincolor}{%
  \begin{minipage}{\linewidth}
    \noindent
    \textbf{Stagiaire informaticien}\hfill 02/2021 - 06/2021\\
    Numerika – Baie-Mahault\\[-0.3em]
    \begin{itemize}[leftmargin=*]
      \item Installer et entretenir les équipements informatiques du parc client. \item Assurer un support utilisateur de premier niveau, améliorant la réactivité du service.
    \end{itemize}
  \end{minipage}}

\medskip\fullrule

% ---------- Éducation -------------------------------------------
\cvsection{Éducation}
\hspace*{1.3cm}%

\begin{tabularx}{\linewidth}{@{}c  >{\RaggedRight\arraybackslash}X
                             >{\raggedleft\arraybackslash}p{0.25\linewidth}@{}}
\textcolor{sidetext}{\faGraduationCap} &
Bachelor Marketing digital &
09/2023 - 08/2024 \\
& CFA IUTS & \\   % ligne de l’établissement
\end{tabularx}
\begin{itemize}[leftmargin=*]
  \item SEO, SEA et gestion des réseaux sociaux.
  \item Pilotage de projets digitaux et analyse de performance.
  \item Création de contenu, campagnes e-mailing, reporting.
\end{itemize}
\vspace{3mm}

\begin{tabularx}{\linewidth}{@{}c  >{\RaggedRight\arraybackslash}X
                             >{\raggedleft\arraybackslash}p{0.25\linewidth}@{}}
\textcolor{sidetext}{\faGraduationCap} &
BTS Systèmes numériques option informatique et réseaux &
09/2019 - 06/2021 \\
& Lycée Chevalier Saint-Georges – Les Abymes & \\   % ligne de l’établissement
\end{tabularx}
\begin{itemize}[leftmargin=*]
  \item Architecture réseau et systèmes.
  \item Maintenance et support technique.
  \item Programmation et configuration d’équipements connectés.
\end{itemize}

\medskip\fullrule

% ---------- Compétences -----------------------------------------
\cvsection{Compétences}

\begin{tabular}{@{}p{0.25\linewidth}p{0.18\linewidth}p{0.18\linewidth}p{0.18\linewidth}}\cicon Administration & \cicon Réseaux & \cicon Support & \cicon Maintenance \\
\cicon Diagnostic & \cicon Installation & \cicon Marketing & \cicon Configuration \\\end{tabular}   % grille 3 lignes × 4 colonnes

\end{document}

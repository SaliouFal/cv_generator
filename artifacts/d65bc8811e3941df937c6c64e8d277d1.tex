\documentclass[a4paper]{article}

%%%%%%%%%%%%%%%%%%%%%%%%%%%%%%%%%%%%%%%%%%%%%%
\usepackage[T1]{fontenc}
\usepackage{geometry}
\geometry{a4paper,left=1.5cm,right=1cm,top=1cm,bottom=1cm}

\usepackage{graphicx}
\usepackage[absolute,overlay]{textpos}
\usepackage{eso-pic}               % image de fond
\usepackage{fontawesome5}
\usepackage[hidelinks]{hyperref}
\usepackage{tikz}
\usepackage{xcolor}
\usepackage{enumitem}
\setlist{nosep,leftmargin=6mm}
\usepackage{times}                % même police que votre exemple
\usepackage{array} 
\usepackage{tabularx}
\usepackage{ragged2e}
\let\origcolorbox\colorbox    % sauvegarde
\renewcommand{\colorbox}[2]{#2}% neutralise le fond
%%%%%%%%%%%%%%%%%%%%%%%%%%%%%%%%%%%%%%%%%%%%%%
%\definecolor{texcolor}{HTML}{e2e8f0}
\providecolor{sidetext}{rgb}{1,1,1}
\definecolor{maincolor}{HTML}{ffffff}

%%%%%%%%%%%%%%%%%%%%%%%%%%%%%%%%%%%%%%%%%
% — Ne changez pas le nom : « background.jpg » doit être présent
\AddToShipoutPictureBG*{%
  \includegraphics[width=\paperwidth,height=\paperheight]{background.jpg}%
}

%%%%%%%%%%%%%%%%%%%%%%%%%%%%%%%%%%%%%%%%%
\newcommand{\fullrule}{\hspace{-1.5cm}\rule{\paperwidth}{0.4pt}}
\newcommand{\cvsection}[1]{%
  \vspace{6pt}\textbf{\Large #1}\par\vspace{2pt}}
\newcommand{\cicon}[1]{%
  \tikz[baseline]{\draw[fill=white] (0,0.1) circle[radius=0.1cm];}~#1}

\setlength{\parindent}{0pt}
%\color{texcolor}
%%%%%%%%%%%%%%%%%%%%%%%%%%%%%%%%%%%%%%%%%%%%%%%%%%%%%%%%%%%%%%
\begin{document}
\color{white}
% ---------- Photo ------------------------------------------------
\begin{textblock*}{4cm}(0.2cm,0.3cm)
  \includegraphics[width=2.5cm,clip,keepaspectratio]{2f07b9dc2a434c869b9123b2075fa08d.png}
\end{textblock*}

% ---------- En-tête ---------------------------------------------
\begin{center}
  {\fontsize{44pt}{24pt}\selectfont\bfseries Judikael Mourouvin}

  \bigskip
  {\Large Technicien support informatique \& marketing digital}

  \bigskip\bigskip
  \faMapMarker~Route de Cocoyer\ 97190 Gosier
  \quad\faEnvelope~\href{mailto:jkmou971@gmail.com}{jkmou971@gmail.com}

  \bigskip
  % Badge LinkedIn (retirez-le si inutile)
  \faPhone~ +590 0690 91 14 48
  \quad \faLinkedin\ \href{}{}
 

  \vspace{-0.3cm}
  \medskip\fullrule
\end{center}

% ---------- Profil ----------------------------------------------
\cvsection{Profil}

Technicien orienté marketing digital, je combine compétences en support informatique et communication en ligne. Mon année d’alternance à la DSI de la Mairie du Gosier m’a permis de gérer des projets numériques et d’accompagner les utilisateurs au quotidien. Reconnu pour ma rigueur et ma capacité d’analyse, je sais diagnostiquer rapidement les incidents et proposer des solutions adaptées. Je souhaite désormais mettre mon savoir-faire au service de nouvelles équipes et contribuer à la réussite de leurs initiatives digitales.

\medskip\fullrule

% ---------- Expérience ------------------------------------------
\cvsection{Expérience}
\hspace*{1.3cm}%

\colorbox{maincolor}{%
  \begin{minipage}{\linewidth}
    \textbf{Alternant en Marketing Digital} \\ Mairie du Gosier - DSI \\ 2023-2024
    \begin{itemize}
      \item Géré des projets numériques municipaux, assurant leur mise en œuvre conforme aux objectifs. \item Analysé les besoins des agents puis déployé des solutions adaptées pour optimiser leurs processus. \item Assuré le support technique et animé des sessions de formation, renforçant la stratégie digitale de la ville.
    \end{itemize}
  \end{minipage}}

\vspace{3mm}


\colorbox{maincolor}{%
  \begin{minipage}{\linewidth}
    \textbf{Animateur de la zone informatique} \\ Pôle Emploi, Gosier \\ 2022-2023
    \begin{itemize}
      \item Fournit un support quotidien aux usagers pour garantir la disponibilité des postes en libre-service. \item Configuré, entretenu et sécurisé les équipements afin de maintenir un parc fonctionnel. \item Diagnostiqué et résolu les incidents matériels ou réseaux, réduisant les interruptions de service.
    \end{itemize}
  \end{minipage}}

\vspace{3mm}


\colorbox{maincolor}{%
  \begin{minipage}{\linewidth}
    \textbf{Stagiaire Informaticien} \\ Numerika, Baie-Mahault \\ 2020-2021
    \begin{itemize}
      \item Installé et paramétré postes et périphériques pour assurer leur conformité aux standards internes. \item Réalisé la maintenance préventive et corrective du parc, limitant les pannes. \item Assuré un support de proximité aux utilisateurs, améliorant la satisfaction.
    \end{itemize}
  \end{minipage}}

\medskip\fullrule

% ---------- Éducation -------------------------------------------
\cvsection{Éducation}
\hspace*{1.3cm}%

    \begin{tabularx}{\linewidth}{@{}c >{\RaggedRight\arraybackslash}X@{}}
    \textcolor{sidetext}{\faGraduationCap} &
    \textbf{Bachelor Marketing Digital} \\
    & CFA IUTS \\
    & \textit{2023-2024} \\
    \end{tabularx}
    \begin{itemize}[leftmargin=*]
  \item Conception de stratégies de contenu et référencement SEO/SEA.
  \item Analyse de données marketing pour optimiser les campagnes.
  \item Gestion de projets et communication multicanale.
\end{itemize}
\vspace{3mm}

    \begin{tabularx}{\linewidth}{@{}c >{\RaggedRight\arraybackslash}X@{}}
    \textcolor{sidetext}{\faGraduationCap} &
    \textbf{BTS Systèmes Numériques option Informatique et Réseaux} \\
    & Lycée de Chevalier Saint Georges, Abymes \\
    & \textit{2019-2021} \\
    \end{tabularx}
    \begin{itemize}[leftmargin=*]
  \item Administration et sécurisation de réseaux et systèmes.
  \item Programmation et intégration de solutions numériques.
  \item Support technique et maintien en conditions opérationnelles.
\end{itemize}

\medskip\fullrule

% ---------- Compétences -----------------------------------------
\cvsection{Compétences}

\hspace*{2cm}%
\begin{tabular}{@{}p{0.25\linewidth}p{0.18\linewidth}p{0.18\linewidth}p{0.18\linewidth}}\cicon Marketing & \cicon Digital & \cicon Réseaux & \cicon Assistance \\
\cicon Maintenance & \cicon Support & \cicon Configuration & \cicon Diagnostic \\
\cicon Administration & \cicon Formation & ~ & ~ \\\end{tabular}   % grille 3 lignes × 4 colonnes

\end{document}

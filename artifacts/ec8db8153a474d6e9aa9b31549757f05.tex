\documentclass{article}

\usepackage{times}
\usepackage{geometry}
\geometry{a4paper,left=0.6cm,right=0.7cm,top=1.5cm,bottom=1cm,columnsep=0.8cm}

\usepackage{fontawesome}          % icônes de base seulement
\usepackage[hidelinks]{hyperref}
\usepackage{multicol}
\usepackage{tikz}
\usepackage{hyphsubst}
\usepackage{moresize}
\usepackage{hyphenat}
\usepackage{tabularx}
\usepackage{xcolor}
\usepackage{enumitem}
\usetikzlibrary{calc, positioning}
\newcolumntype{Y}{>{\RaggedRight\arraybackslash}X}

% icônes manquantes -> puce
\makeatletter
\@for\sym:=faBrain,faMicrochip,faHandshakeO,faTools,faNetworkWired,%
             faDatabase,faServer,faGit,faUsers,faComments,faCalendar,faGroup\do{%
  \@ifundefined{\sym}{\expandafter\newcommand\csname\sym\endcsname{\textbullet}}{}}
\makeatother

% couleurs
\definecolor{maincolor}{HTML}{f0fafc}
\definecolor{seccolor}{HTML}{ffffff}
\definecolor{gray}{HTML}{8c94a9}
\definecolor{sidetext}{HTML}{59cee5}

% bande latérale bleue
\usepackage{eso-pic}
\AddToShipoutPictureBG{%
  \begin{tikzpicture}[remember picture,overlay]
    \fill[maincolor] (current page.north west) rectangle
                     ([xshift=0.3\paperwidth] current page.south west);
  \end{tikzpicture}%
}

% listes
\setlist[itemize]{itemsep=-2pt,topsep=0pt,leftmargin=1.08cm}
\renewcommand{\labelitemi}{\textcolor{sidetext}{\footnotesize$\bullet$}}

\setlength{\parindent}{0pt}
\usepackage{paracol}
\columnratio{0.3}

\begin{document}
\pagestyle{empty}

\begin{paracol}{2}
% ────────────────────────────────────────
% Colonne gauche
% ────────────────────────────────────────
\color{sidetext}
\vspace*{-0.5cm}

\noindent
\begin{minipage}{\linewidth}
  \centering
  \begin{tikzpicture}
    \clip (0,0) circle (1.5cm) node[anchor=center]
      {\includegraphics[width=3cm]{07f4e43d82b14360980f268ad90ca3fa.png}};
  \end{tikzpicture}

  \vspace{3mm}
  {\color{black}\LARGE \textbf{Pape Saliou Fall}}

  \vspace{1mm}
  {\large Ingénieur Data Scientist \& Développeur IA}

  \vspace{3mm}
  {\color{gray}\rule{\linewidth}{0.4pt}} \\
\end{minipage}

% ── Coordonnées
\begin{tabular}{@{}c l}
  \faPhone &
  \begin{tabular}[t]{@{}l@{}}
    {\color{gray}Téléphone} \\ 0753481453
  \end{tabular} \\
  \\
  \faLinkedin &
  \begin{tabular}[t]{@{}l@{}}
    {\color{gray}LinkedIn} \\
    \href{@pape-saliou-fall-43154a211/}{Mon LinkedIn}
  \end{tabular} \\
  \\
  \faMapMarker &
  \begin{tabular}[t]{@{}l@{}}
    {\color{gray}Adresse} \\ 95300 Pontoise \\ 
  \end{tabular} \\
  \\
  \faEnvelope &
  \begin{tabular}[t]{@{}l@{}}
    {\color{gray}Email} \\
    \href{mailto:papesalioufall2@gmail.com}{papesalioufall2@gmail.com}
  \end{tabular} \\
\end{tabular}

\vspace{2mm}
{\color{gray}\rule{\linewidth}{0.4pt}} \\

% ── Langues --------------------------------------------------------
{\color{black}{Langues}}

\vspace{2mm}
\begin{itemize}[leftmargin=*]
\item French - \textcolor{gray}{Native}
\item English - \textcolor{gray}{B2}\end{itemize}          % ← le placeholder va contenir \begin{itemize}…\end{itemize}

{\color{gray}\rule{\linewidth}{0.4pt}} \\

% ── Compétences ----------------------------------------------------
\vspace{2mm}
{\color{black}{Compétences Clés}}

\vspace{2mm}
\begin{itemize}[leftmargin=*]
\item Python
\item JavaScript
\item SQL
\item PowerBI
\item Git
\item GitHub
\item TensorFlow\end{itemize}              % ← idem, une vraie liste
\vspace{2mm}
{\color{gray}\rule{\linewidth}{0.4pt}} \\

% ── Centres d'intérêt
\vspace{2mm}
{\color{black}{Centres d’intérêt}}

\vspace{2mm}
\begin{itemize}[leftmargin=*]
\item Football
\item Natation
\item Lecture
\end{itemize}     % ← simple itemize ou tabular

\vfill
~

% ────────────────────────────────────────
\switchcolumn
% Colonne droite
% ────────────────────────────────────────
\color{black}

% ── Profil
\textcolor{black}{\Large \textbf{Profil Professionnel}} \\[2pt]
Data Scientist et développeur IA passionné, je conçois des solutions innovantes à partir de données complexes pour répondre à des enjeux métier variés. Autonome et proactif, j’ai déjà mené des projets de machine learning, de deep learning et d’analyse de séries temporelles jusqu’à leur mise en production. J’apprécie le travail d’équipe et la transmission de bonnes pratiques pour garantir la qualité du code et la fiabilité des modèles. Désireux de nouveaux défis, je recherche un environnement dynamique où l’innovation et l’excellence sont au cœur des projets. \\[8pt]

% ── Expérience
\textcolor{black}{\Large \textbf{Expérience Professionnelle}} \\[2pt]

\colorbox{maincolor}{%
  \begin{minipage}{\linewidth}
    \textbf{Data Scientist \& Développeur IA} \\ Prepaya \\ 01/2024 – Présent
    \begin{itemize}
      \item Développé une plateforme IA en Python/JavaScript déployée sur Heroku, améliorant la performance applicative. \item Mené des analyses de séries temporelles et des modèles ML/DL pour accroître la précision prédictive des rapports. \item Intégré des API OpenAI et bases PostgreSQL via Flask, assurant une mise en production rapide et fiable.
    \end{itemize}
  \end{minipage}}

\vspace{3mm}


\colorbox{maincolor}{%
  \begin{minipage}{\linewidth}
    \textbf{Apprenti Risk Analyst \& Data Scientist} \\ AXA XL (Groupe AXA) \\ 12/2022 – 12/2023
    \begin{itemize}
      \item Automatisé la collecte des données financières, réduisant le temps de traitement de 40 \%. \item Créé des tableaux de bord Power BI et Excel pour la facturation, améliorant la visibilité pour finance et management. \item Développé des applications prédictives sur les sinistres (Python, R, SQL) pour estimer la probabilité de survenance.
    \end{itemize}
  \end{minipage}}

\vspace{3mm}


\colorbox{maincolor}{%
  \begin{minipage}{\linewidth}
    \textbf{Apprenti Data Scientist} \\ Prepaya \\ 09/2021 – 08/2022
    \begin{itemize}
      \item Implémenté des modèles NLP et deep learning pour générer automatiquement des formulaires clients. \item Réalisé une analyse de sentiments des retours utilisateurs, fournissant des insights actionnables au support. \item Utilisé Python (BeautifulSoup, Selenium, PyTorch) et BERT/T5 pour optimiser le pipeline de traitement de texte.
    \end{itemize}
  \end{minipage}}   % ← blocs \colorbox{maincolor}{\begin{minipage}…}

\vspace{8mm}

% ── Formation
\textcolor{black}{\Large \textbf{Formation}} \\[2pt]

    \begin{tabularx}{\linewidth}{@{}c X@{}}
    \textcolor{sidetext}{\faGraduationCap} &
    \textbf{Master 2 Data Science} \\
    & Sorbonne Université \\
    & \begin{itemize}[leftmargin=*]
  \item Approfondissement en analyse de données, machine learning et deep learning. \item Cours sur modèles de structure latente, séries chronologiques et apprentissage statistique. \item Travaux pratiques en bases de données et calcul parallèle pour projets à grande échelle.
\end{itemize} \\
    & \textit{09/2021 – 03/2022}
    \end{tabularx}
           % ← lignes tabular par diplôme

\end{paracol}
\end{document}


\documentclass[a4paper]{article}

%%%%%%%%%%%%%%%%%%%%%%%%%  PACKAGES  %%%%%%%%%%%%%%%%%%%%%%%%%
\usepackage[T1]{fontenc}
\usepackage{geometry}
\geometry{a4paper,left=1.5cm,right=1cm,top=1cm,bottom=1cm}
\usepackage[utf8]{inputenc}
\usepackage{graphicx}
\usepackage[absolute,overlay]{textpos}
% \usepackage{eso-pic}               % image de fond – supprimé car plus utilisé
\usepackage{textcomp}
\usepackage{newtxtext}
\usepackage{fontawesome5}
\usepackage[hidelinks]{hyperref}
\usepackage{tikz}
\usepackage{xcolor}
\usepackage{enumitem}
\setlist{nosep,leftmargin=6mm}
\usepackage{times}                % même police que votre exemple
\usepackage{array}
\usepackage{tabularx}
\usepackage{ragged2e}
\let\origcolorbox\colorbox    % sauvegarde
\renewcommand{\colorbox}[2]{#2}% neutralise le fond

%%%%%%%%%%%%%%%%%%%%%%%%%  COULEURS  %%%%%%%%%%%%%%%%%%%%%%%%%
% Définition d’un gris très clair (~95 % blanc)
\definecolor{verylightgray}{HTML}{F5F5F5}

%%%%%%%%%%%%%%%%%%%%%%%%%  PAGE & TEXTE  %%%%%%%%%%%%%%%%%%%%%
\pagecolor{verylightgray}   % fond gris très clair
% le texte est noir par défaut ; aucune redéfinition nécessaire

%%%%%%%%%%%%%%%%%%%%%%%%%  COULEURS  %%%%%%%%%%%%%%%%%%%%%%%%%
%\pagecolor{lightgray}     % fond gris clair
\color{black}             % texte en noir

% Vous pouvez ajuster la nuance si besoin : par ex. \pagecolor{gray!15}

\providecolor{sidetext}{rgb}{1,1,1}
\definecolor{maincolor}{HTML}{ffffff}

%%%%%%%%%%%%%%%%%%%%%  MACROS UTILES  %%%%%%%%%%%%%%%%%%%%%%%
\newcommand{\fullrule}{\hspace{-1.5cm}\rule{\paperwidth}{0.4pt}}
\newcommand{\cvsection}[1]{%
  \vspace{6pt}\textbf{\Large #1}\par\vspace{2pt}}
\newcommand{\cicon}[1]{%
  \tikz[baseline]{\draw[fill=white] (0,0.1) circle[radius=0.1cm];}~#1}

\setlength{\parindent}{0pt}
%%%%%%%%%%%%%%%%%%%%%%%%%%%%%%%%%%%%%%%%%%%%%%%%%%%%%%%%%%%%%%
\begin{document}

% ---------- Photo ------------------------------------------------
\ifx\relax%%PHOTO_FILE%%\relax\else
\begin{textblock*}{4cm}(0.2cm,0.3cm)
  \includegraphics[width=2.5cm,clip,keepaspectratio]{%%PHOTO_FILE%%}
\end{textblock*}
 \fi
% ---------- En-tête ---------------------------------------------
\begin{center}
  {\fontsize{44pt}{24pt}\selectfont\bfseries %%FULL_NAME%%}

  \bigskip
  {\Large %%HEADLINE%%}

  \bigskip\bigskip
  \faMapMarker~%%ADDRESS_LINE1%%\ %%ADDRESS_LINE2%%
  \quad\faEnvelope~\href{mailto:%%EMAIL%%}{%%EMAIL%%}

  \bigskip
  % Badge LinkedIn (retirez-le si inutile)
  \faPhone~ %%PHONE%%
  \quad \faLinkedin\ \href{%%LINKEDIN_URL%%}{%%LINKEDIN_HANDLE%%}
 

  %\vspace{-0.3cm}
  
\end{center}
%\medskip\fullrule
 \vspace{0.6cm}
% ---------- Profil ----------------------------------------------
\cvsection{Profil}
\vspace{0.3cm}
%%SUMMARY%%

\medskip\fullrule

% ---------- Expérience ------------------------------------------
\cvsection{Expérience}
\vspace{0.3cm}
%%EXPERIENCE_BLOCKS%%

\medskip\fullrule

% ---------- Éducation -------------------------------------------
\cvsection{Éducation}
\vspace{0.3cm}

%%DEGREE_BLOCKS%%

\medskip\fullrule

% ---------- Compétences -----------------------------------------
\cvsection{Compétences}
\vspace{0.3cm}

%%SKILL_CIRCLES%%   % grille 3 lignes × 4 colonnes

\end{document}

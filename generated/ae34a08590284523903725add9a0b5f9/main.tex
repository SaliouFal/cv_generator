% !TEX program = xelatex
% -------------------------------------------------------------
%  Clean CV template — Letter size (8.5 × 11 in)
%  Matches the blueprint supplied by the user.
%  Compile with **XeLaTeX** on Overleaf (Menu ▸ Compiler ▸ XeLaTeX)
% -------------------------------------------------------------

%=================================================================
%                PREAMBLE
%=================================================================
\documentclass[10pt,letterpaper]{article}

% ---------- ENCODING / FONTS ------------------------------------
\usepackage{fontspec}
\IfFontExistsTF{Roboto}
  {\setmainfont{Roboto}}
  {\setmainfont{Helvetica Neue}}

% ---------- GEOMETRY --------------------------------------------
\usepackage[
  letterpaper,
  left=0.55in,
  right=0.55in,
  top=0.8in,
  bottom=0.8in
]{geometry}

% ---------- COLOURS & GRAPHICS ----------------------------------
\usepackage[dvipsnames,svgnames,x11names]{xcolor}
\definecolor{primary}{HTML}{004A99}
\definecolor{accent}{HTML}{E6F4FF}

\usepackage{graphicx}
\usepackage{tikz}
\usetikzlibrary{calc}

% ---------- LAYOUT HELPERS --------------------------------------
\usepackage{paracol}
\columnratio{0.32}
\setlength{\columnsep}{0.25in}

\usepackage[most]{tcolorbox}
\tcbset{colback=accent, colframe=accent, boxrule=0pt, sharp corners}

\usepackage{enumitem}
\setlist[itemize]{noitemsep,topsep=0pt,leftmargin=*}

% ---------- HEADING STYLES --------------------------------------
\usepackage{sectsty}
\allsectionsfont{\color{primary}\bfseries\uppercase}
\subsectionfont{\color{primary}\bfseries}
\renewcommand{\thesection}{}

% ---------- UTILITIES -------------------------------------------
\newcommand{\cvName}[1]{\vspace*{0.3in}\textbf{\LARGE #1}}
\newcommand{\cvHeadline}[1]{\par\smallskip\textit{#1}}
\newcommand{\cvHr}{\vspace{0.5\baselineskip}\hrule height 1pt\color{primary}\vspace{0.7\baselineskip}}

%=================================================================
%                DOCUMENT
%=================================================================
\begin{document}

% -------------------- TWO–COLUMN LAYOUT -------------------------
\begin{paracol}{2}

% -------- LEFT SIDEBAR ------------------------------------------
\begin{leftcolumn}
\begin{center}
% --- Avatar -----------------------------------------------------
\begin{tikzpicture}
  \node[draw=primary,line width=1pt,circle,minimum width=1.6in,minimum height=1.6in]  (photo) {\includegraphics[width=1.5in]{}};
\end{tikzpicture}
\end{center}

\vspace{0.6in}

% --- Name & headline -------------------------------------------
\cvName{Pape FALL}
\cvHeadline{Data Scientist}

\cvHr

% --- Contact Information ---------------------------------------
\section*{Contact Information}
0753481453\\
pape@gmail.com\\
\textit{LinkedIn :} \\
12 rue des Data, 75013 Paris, France\\
% (aucun site web communiqué)

\cvHr

% --- Languages --------------------------------------------------
\section*{Languages}
Français\\
Anglais

\cvHr

% --- Key Skills -------------------------------------------------
\section*{Key Skills}
\begin{itemize}
  \item Python
  \item SQL
  \item Pandas / NumPy
  \item Scikit-learn
  \item TensorFlow / Keras
  \item Machine Learning (supervisé \& non supervisé)
\end{itemize}

\cvHr

% --- Hobbies ----------------------------------------------------
\section*{Hobbies}
Musique jazz, Football, Échecs, Photographie urbaine

\end{leftcolumn}

% -------- MAIN COLUMN ------------------------------------------
\begin{rightcolumn}

% --- Professional Summary --------------------------------------
\section*{Professional Summary}
Data Scientist avec plus de 2 ans d’expérience dans l’exploitation des données pour créer des produits analytiques à forte valeur ajoutée. Expert en Python, machine learning et visualisation, je conçois des modèles prédictifs robustes, optimise les pipelines de données et communique des insights clairs aux parties prenantes. Passionné par la recherche de solutions innovantes, je combine rigueur scientifique et esprit business pour accélérer la prise de décision.

\vspace{1in}

% --- Work Experience -------------------------------------------
\section*{Work Experience}

% ----- Experience block 1 --------------------------------------
\begin{tcolorbox}
  \begin{minipage}[t]{0.48\linewidth}
    \textbf{Data Scientist — Prepaya}\\
    Paris, France
    \begin{itemize}
      \item Développé un modèle de scoring client qui a augmenté le taux de conversion de 18 \%.
      \item Implémenté des pipelines ETL sur Airflow pour automatiser la collecte de données (+4 h économisées/jour).
      \item Mise en place d’une architecture MLOps (MLflow, Docker, GitHub Actions) réduisant de 30 \% le temps de déploiement.
      \item Collaboré avec les équipes produit et marketing pour transformer les besoins business en solutions data.
      \item Présenté les résultats à la direction via des dashboards Power BI interactifs.
    \end{itemize}
  \end{minipage}\hfill
  \begin{minipage}[t]{0.48\linewidth}
    \raggedleft
    % dates non fournies
  \end{minipage}
\end{tcolorbox}

% ----- Experience block 2 --------------------------------------
\begin{tcolorbox}
  \begin{minipage}[t]{0.48\linewidth}
    \textbf{Stagiaire Data Analyst — EDF}\\
    Nanterre, France
    \begin{itemize}
      \item Analyse de séries temporelles de consommation électrique pour prédire les pics de demande (MAPE < 5 \%).
      \item Création d’un modèle de classification d’anomalies permettant une détection 40 \% plus rapide des incidents réseau.
      \item Automatisation de rapports Power BI consultés par plus de 70 utilisateurs internes.
      \item Nettoyage et enrichissement de jeux de données brutes (Spark, SQL) de plus de 200 M de lignes.
      \item Contribution à la stratégie Data \& IA du département R\&D.
    \end{itemize}
  \end{minipage}\hfill
  \begin{minipage}[t]{0.48\linewidth}
    \raggedleft
    % dates non fournies
  \end{minipage}
\end{tcolorbox}

\vspace{0.9in}

% --- Education --------------------------------------------------
\section*{Education}
\begin{tcolorbox}[colback=white,boxrule=1pt,colframe=primary]
  \begin{minipage}{0.47\linewidth}
    Master 2 Data Science\\
    Sorbonne Université\\
    2022–2024
  \end{minipage}
\end{tcolorbox}

% --- Certifications --------------------------------------------
\section*{Certifications}
\begin{itemize}
  \item TensorFlow Developer Certificate — Google (Mar-2023)
  \item Azure AI Fundamentals — Microsoft (Jul-2022)
\end{itemize}

\end{rightcolumn}

\end{paracol}

\end{document}